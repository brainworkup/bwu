\documentclass[11pt]{article}
\usepackage[textwidth=5.5in,textheight=8.5in]{geometry}
\usepackage[T1]{fontenc}
\usepackage[osf,proportional]{libertine}
\usepackage{lettrine}
\renewcommand{\ttfamily}{\fontencoding{OT1}\fontfamily{cmtt}\selectfont}
\PassOptionsToPackage{urlcolor=black,colorlinks}{hyperref}
\RequirePackage{hyperref}
\usepackage{xcolor}
\usepackage{longtable}
\usepackage{multicol}
\newcommand{\typeone}{Type~\liningnums{1}}
\begin{document}
\title{\LaTeX\ Support for Linux Libertine and Biolinum Fonts}
\author{Bob Tennent\\
\small\url{rdt@cs.queensu.ca}}
\date{\today}
\maketitle
\thispagestyle{empty}
\begin{small}
\tableofcontents
\end{small}
\sloppy
\clearpage
\section{Introduction}

This package provides support for use of the Linux Libertine and Linux
Biolinum families of fonts in \LaTeX.
Most features
are usable
with \LaTeX\ and \verb|dvips|, pdf\LaTeX, xe\LaTeX\ and lua\LaTeX;
the features in Section~\ref{OpenType} are only usable with xe\LaTeX\ or lua\LaTeX.
This package compatibly replaces several earlier packages (\texttt{libertine-type1},
\texttt{biolinum-type1}, \texttt{libertine}) and should provide partial compatibility with
the obsolete \texttt{libertineotf} and \texttt{libertine-legacy} packages.

\section{Installation}

To install this package on a TDS-compliant \TeX\ system, download the
file
\begin{list}{}{}\item \verb\tex-archive/install/fonts/libertine.tds.zip\
\end{list}
and unzip at the root of an appropriate
\verb\texmf\ tree, likely a personal or local tree. If necessary, update
the file-name database (e.g., \verb\texhash texmf\). Update the font-map files by
enabling the Map file \verb\libertine.map\.


\section{Basic Usage}

For most purposes, simply add

\begin{list}{}{}\item
\verb|\usepackage{libertine}|
\end{list}
to the preamble of your document. This will activate Libertine as
the main (seriffed) text font, Biolinum as the sans font,
and (from January~2013) LibertineMono as the monospaced font.
It is
recommended that the font encoding be set to \verb\T1\ or \verb\LY1\ but the default
\verb\OT1\  encoding is also supported. Available shapes in all series (except \texttt{tt}, which
only has \texttt{it}) include:
\begin{list}{}{}\item
\begin{tabular}{ll}
\texttt{it} &               italic\\
\texttt{sc} &            small caps\\
\texttt{scit} &            italic small caps
\end{tabular}
\end{list}
Slanted variants are not supported; the designed italic variants will be
automatically substituted. The exceptions are the monospaced font and the bold series of Biolinum,
for which designed italics are not currently available. Artificially
slanted variants have been generated and treated as if they were italic.

To activate Libertine (without Biolinum), use the \texttt{libertine} (or \texttt{rm})
option. Similarly, to activate Biolinum (without Libertine) use the
\texttt{biolinum} (or \texttt{sf} or \texttt{ss}) option. To use Biolinum as the main text font (as
well as the sans font), use the option \texttt{sfdefault}.
Use the \verb|mono=false| (or \verb|tt=false|) option to suppress
activating LibertineMono.
To activate single font families,
use one or more of
\begin{list}{}{}
\item \verb|\usepackage{libertineRoman}|
\item \verb|\usepackage{libertineMono}|
\item \verb|\usepackage{biolinum}|
\end{list}


\section{Advanced Usage}

Lua\LaTeX\ and xe\LaTeX\ users who might prefer to use Type~1 fonts or who
wish to avoid \texttt{fontspec} may use the \texttt{type1} (or \texttt{nofontspec}) option.
The \verb\libertine-type1.sty\,
\verb\biolinum-type1.sty\ and \verb\libertineMono-type1.sty\ packages provide
compatibility
with older packages. For legacy documents that use only basic
facilities of \verb\libertineotf\, a wrapper package \verb\libertineotf.sty\ is provided.
The following features of the original \verb|libertine| or \verb|libertineotf|
packages are
\emph{not} supported:
\begin{itemize}

 \item font-features such as \texttt{Ligatures} or \texttt{Scale} as option parameters

 \item the Outline or Shadow fonts

 \item commands \verb|\Lnnum|, \verb|\Lpnum|, \verb|\Lcnum|, etc.

 \item environments \texttt{Ltable} and \texttt{libertineenumerate}
\end{itemize}
If your documents use any of the features listed above,
you may have to continue to use the \verb\libertineotf\ package (which is still available from
CTAN) or access the OpenType fonts directly using \texttt{fontspec}.


The following options are available in all styles (except monospaced):
\begin{list}{}{}\item
\begin{tabular}{ll}
\texttt{oldstyle} (\texttt{osf}) &  old-style figures\\
\texttt{lining} (\texttt{nf}, \texttt{lf}) &  lining figures\\
\texttt{proportional} (\texttt{p})&  varying-width figures\\
\texttt{tabular} (\texttt{t}) &     fixed-width figures\\
\end{tabular}
\end{list}
The defaults (from January~2013) are \texttt{lining} and \texttt{tabular}. These apply to both Libertine
and Biolinum;
to change the default figure style of just the Biolinum (sans)
fonts, use options
\begin{list}{}{}\item
\texttt{sflining} (\texttt{sflf}) or \texttt{sfoldstyle} (\texttt{sfosf}, \texttt{osfss})
\item
\texttt{sftabular} (\texttt{sft}) or \texttt{sfproportional} (\texttt{sfp})
\end{list}

The \texttt{semibold} (\texttt{sb}) option will enable use of the
semi-bold series of Libertine; this has no effect on the Biolinum fonts,
for which there is no semi-bold variant. The options \verb|scale=|<\emph{number}> (or
\verb|scaled=|<\emph{number}>) will scale the Biolinum fonts but have no effect on the
Libertine fonts. Similarly, the options
\verb|llscale=|<\emph{number}> (or  \verb|llscaled=|<\emph{number}>)
and
\verb|ttscale=|<\emph{number}> (or  \verb|ttscaled=|<\emph{number}>)
will scale the LinuxLibertine and LibertineMono fonts, respectively.
Any of the ``Boolean'' options, such as \texttt{osf}, may
also be used
in the form \verb|osf=true| or \verb|osf=false|.

The option \verb\defaultfeatures=...\ allows the user to add default OpenType
features; for example, \verb\defaultfeatures={Variant=01}\ will force use of the Stylistic~Set~1
variant glyphs.

Commands \verb|\oldstylenums{|\ldots\verb|}| and \verb|\oldstylenumsf{|\ldots\verb|}| are defined to
allow for local use of old-style figures in Libertine and Biolinum,
respectively, if lining figures is the default, and similarly
\verb|\liningnums{|\ldots\verb|}| and \verb|\liningnumsf{|\ldots\verb|}|.

Similarly, commands \verb|\tabularnums{|\ldots\verb|}| and \verb|\tabularnumsf{|\ldots\verb|}| are defined
to allow local use of monospaced figures in Libertine or Biolinum,
respectively, if proportional figures is the default, and similarly
\verb|\proportionalnums{|\ldots\verb|}| and \verb|\proportionalnumsf{|\ldots\verb|}|.

Superior numbers (for footnote markers) are available using \verb|\sufigures|
or \verb|\textsu{|\ldots\verb|}|.


Command \verb|\useosf| switches the default figure style for Libertine and Biolinum to old-style figures; this is
primarily for use \emph{after} calling a math package (such as \verb|newtxmath| with the
\verb|libertine| option) with lining figures as the default.

The following macros select the font family indicated:
\begin{center}
\begin{tabular}{ll}
\verb|\libertine| & Libertine \\
\verb|\libertineSB|& Libertine with semibold \\
\verb|\libertineOsF| & Libertine with oldstyle figures \\
\verb|\libertineLF| & Libertine with lining figures \\
\verb|\libertineDisplay| & Libertine Display \\
\verb|\libmono| & Libertine Monospaced \\
\verb|\libertineInitial| & Libertine Initials \\
\verb|\biolinum| & Biolinum \\
\verb|\biolinumOsF|& Biolinum with oldstyle figures \\
\verb|\biolinumLF| & Biolinum with lining figures \\
\end{tabular}
\end{center}
Macro \verb|\libertineInitialGlyph{|\ldots\verb|}| produces a glyph in the Libertine Initial font;
Appendix~\ref{InitialGlyphs} has a table of some of the glyphs.

\section{OpenType Fonts}
\label{OpenType}

The features in this section are only available to xe\LaTeX\ and lua\LaTeX\ users.

Macros \verb|\libertineGlyph{|\ldots\verb|}| and \verb|\biolinumGlyph{|\ldots\verb|}| produce the
glyph named in the argument in the Libertine or Biolinum font,
respectively; for example, in regular-weight and upright-shape,
\verb|\libertineGlyph{seven.cap}| and \verb|\libertineGlyph{uniE10F}| both produce a
lining~7 that matches the height of capital letters, as in
\begin{list}{}{}\item
K\libertineGlyph{seven.cap}L~\libertineGlyph{three.cap}N\libertineGlyph{six.cap}
\end{list}
Similarly, \verb|\biolinumKeyGlyph{|\ldots\verb|}| produces the named glyph
in the Biolinum Keyboard font; for example: \verb|\biolinumKeyGlyph{seven}| produces \biolinumKeyGlyph{seven}.
A large number of macros of the form \verb|\LKey|\ldots or \verb|\LMouse|\ldots
are provided to simplify production of glyphs in the Biolinum Keyboard font;
see Appendix~\ref{LKey}. Appendix~\ref{KeyboardGlyphs} has a table of the entire
Linux Biolinum Keyboard font, with corresponding glyph name and codepoint.


The directory
\verb|/fonts/opentype/public/libertine|
has the fonts used for these features, as follows:
\begin{list}{}{}\item\small
\begin{tabular}{lll}
\multicolumn{1}{c}{\bf File name} & \multicolumn{1}{c}{\bf Internal name} & \multicolumn{1}{c}{\bf Description} \\
\hline
\verb|LinBiolinum_RBO.otf|  & \verb|LinBiolinumOBO|  & sans serif bold italic (oblique) \\
\verb|LinBiolinum_RB.otf|  & \verb|LinBiolinumOB| & sans serif bold \\
\verb|LinBiolinum_RI.otf|  & \verb|LinBiolinumOI| & sans serif italic \\
\verb|LinBiolinum_R.otf|  & \verb|LinBiolinumO| & sans serif regular \\
\verb|LinLibertine_RBI.otf|  & \verb|LinLibertineOBI| & bold italic \\
\verb|LinLibertine_RB.otf|  & \verb|LinLibertineOB| & bold \\
\verb|LinLibertine_RI.otf|  & \verb|LinLibertineOI| & italic \\
\verb|LinLibertine_R.otf|  & \verb|LinLibertineO| & regular \\
\verb|LinLibertine_RZI.otf|  & \verb|LinLibertineOZI| & semibold italic \\
\verb|LinLibertine_RZ.otf|  & \verb|LinLibertineOZ| & semibold \\
\verb|LinLibertine_MBO.otf| & \verb|LinLibertineMOBO| & mono bold italic (oblique) \\
\verb|LinLibertine_MB.otf| & \verb|LinLibertineMOB| & mono bold \\
\verb|LinLibertine_MO.otf| & \verb|LinLibertineMOO| & mono italic (oblique) \\
\verb|LinLibertine_M.otf| & \verb|LinLibertineMO| & mono \\
\verb|LinBiolinum_K.otf|  & \verb|LinBiolinumOKb| & keyboard \\
\verb|LinLibertine_I.otf|  & \verb|LinLibertineIO| &   decorative capitals \\
\verb|LinLibertine_DR.otf| & \verb|LinLibertineDisplayO| &   a display (titling) font  \\
\end{tabular}
\end{list}

\section{Concluding Remarks}


For compatible mathematics, it is recommended to use
\begin{verbatim}
  \usepackage[libertine]{newtxmath}
\end{verbatim}
with pdf\LaTeX\ and
\begin{verbatim}
  \usepackage{unicode-math}
  \setmathfont[Scale=MatchUppercase]{libertinusmath-regular.otf}
\end{verbatim}
with xe\LaTeX\ or lua\LaTeX.

The original OpenType fonts were created by Philipp H. Poll
(\url{gillian@linuxlibertine.org}) and are licensed under the terms of the GNU General
Public License (Version~2, with font exception) and under the terms of
the Open Font License. For details look into the \verb|doc| directory of the
distribution or at
\begin{list}{}{}\item
\url{http://www.linuxlibertine.org/}
\end{list}
The Glyph and KeyCap support was adapted from the original \verb\libertine\
package by Michael Niedermair.

Three of the Libertine fonts were modified by Michael Sharpe (\url{msharpe@ucsd.edu}) using
\texttt{fontforge} to correct minor problems, including adding
three missing ligatures (\emph{\bfseries fl, ffl, ffi}) to the bold-italic font.

The \typeone\ fonts were created using \verb|cfftot1| or \verb|fontforge|.
The internal font-family names of the \typeone\
fonts have been changed to \verb|Linux Libertine T| and \verb|Linux Biolinum T| to avoid
interfering with xe\LaTeX\ users who access system fonts.

The support
files were created using \verb|autoinst|. The support files are licensed under
the terms of the LaTeX Project Public License.
See Appendix~\ref{impl} for more detailed discussion of the implementation.

Thanks to Herbert Voss, Patrick Gundlach, Silke Hofstra, Marc Penninga, Michael Sharpe,
Denis Bitouz\'{e}, and Khaled Hosny for their assistance.
The maintainer of this
package is Bob Tennent (\url{rdt@cs.queensu.ca})

\clearpage

\appendix

\setlength{\fboxrule}{0.1pt}
\section{Biolinum KeyCap Macros}\small\tt
\label{LKey}

\subsection{Special Keys}
\begin{longtable}[l]{lll}
Tux         & \verb|\LKeyTux|      & \LARGE\strut\fbox{\LKeyTux} \\
Win         & \verb|\LKeyWin|      & \LARGE\strut\fbox{\LKeyWin} \\
Menu        & \verb|\LKeyMenu|     & \LARGE\strut\fbox{\LKeyMenu} \\
Strg        & \verb|\LKeyStrg|     & \LARGE\strut\fbox{\LKeyStrg} \\
Ctrl        & \verb|\LKeyCtrl|     & \LARGE\strut\fbox{\LKeyCtrl} \\
Alt         & \verb|\LKeyAlt|      & \LARGE\strut\fbox{\LKeyAlt} \\
AltGr       & \verb|\LKeyAltGr|    & \LARGE\strut\fbox{\LKeyAltGr} \\
Shift       & \verb|\LKeyShift|    & \LARGE\strut\fbox{\LKeyShift} \\
Enter       & \verb|\LKeyEnter|    & \LARGE\strut\fbox{\LKeyEnter} \\
Tab         & \verb|\LKeyTab|      & \LARGE\strut\fbox{\LKeyTab} \\
CapsLock    & \verb|\LKeyCapsLock| & \LARGE\strut\fbox{\LKeyCapsLock} \\
Pos         & \verb|\LKeyPos|      & \LARGE\strut\fbox{\LKeyPos} \\
Entf        & \verb|\LKeyEntf|     & \LARGE\strut\fbox{\LKeyEntf} \\
Einf        & \verb|\LKeyEinf|     & \LARGE\strut\fbox{\LKeyEinf} \\
Leer        & \verb|\LKeyLeer|     & \LARGE\strut\fbox{\LKeyLeer} \\
Esc         & \verb|\LKeyEsc|      & \LARGE\strut\fbox{\LKeyEsc} \\
Ende        & \verb|\LKeyEnde|     & \LARGE\strut\fbox{\LKeyEnde} \\
Back        & \verb|\LKeyBack|     & \LARGE\strut\fbox{\LKeyBack} \\
Up          & \verb|\LKeyUp|       & \LARGE\strut\fbox{\LKeyUp} \\
Dwon        & \verb|\LKeyDown|     & \LARGE\strut\fbox{\LKeyDown} \\
Left        & \verb|\LKeyLeft|     & \LARGE\strut\fbox{\LKeyLeft} \\
Right       & \verb|\LKeyRight|    & \LARGE\strut\fbox{\LKeyRight} \\
PgUp        & \verb|\LKeyPgUp|     & \LARGE\strut\fbox{\LKeyPgUp} \\
PgDown      & \verb|\LKeyPgDown|   & \LARGE\strut\fbox{\LKeyPgDown} \\
At          & \verb|\LKeyAt|       & \LARGE\strut\fbox{\LKeyAt} \\
Fn          & \verb|\LKeyFn|       & \LARGE\strut\fbox{\LKeyFn} \\
Home        & \verb|\LKeyHome|     & \LARGE\strut\fbox{\LKeyHome} \\
Del         & \verb|\LKeyDel|      & \LARGE\strut\fbox{\LKeyDel} \\
Space       & \verb|\LKeySpace|    & \LARGE\strut\fbox{\LKeySpace} \\
ScreenUp    & \verb|\LKeyScreenUp| & \LARGE\strut\fbox{\LKeyScreenUp} \\
ScreenDown  & \verb|\LKeyScreenDown| & \LARGE\strut\fbox{\LKeyScreenDown} \\
Ins         & \verb|\LKeyIns|      & \LARGE\strut\fbox{\LKeyIns} \\
End         & \verb|\LKeyEnd|      & \LARGE\strut\fbox{\LKeyEnd} \\
GNU         & \verb|\LKeyGNU|      & \LARGE\strut\fbox{\LKeyGNU} \\
PageUp      & \verb|\LKeyPageUp|   & \LARGE\strut\fbox{\LKeyPageUp} \\
PageDown    & \verb|\LKeyPageDown| & \LARGE\strut\fbox{\LKeyPageDown} \\
Command     & \verb|\LKeyCommand| & \LARGE\strut\fbox{\LKeyCommand} \\
OptionKey   & \verb|\LKeyOptionKey| & \LARGE\strut\fbox{\LKeyOptionKey} \\
%
F1          & \verb|\LKeyF{1}|     & \LARGE\strut\fbox{\LKeyF{1}} \\
F2          & \verb|\LKeyF{2}|     & \LARGE\strut\fbox{\LKeyF{2}} \\
F3          & \verb|\LKeyF{3}|     & \LARGE\strut\fbox{\LKeyF{3}} \\
F4          & \verb|\LKeyF{4}|     & \LARGE\strut\fbox{\LKeyF{4}} \\
F5          & \verb|\LKeyF{5}|     & \LARGE\strut\fbox{\LKeyF{5}} \\
F6          & \verb|\LKeyF{6}|     & \LARGE\strut\fbox{\LKeyF{6}} \\
F7          & \verb|\LKeyF{7}|     & \LARGE\strut\fbox{\LKeyF{7}} \\
F8          & \verb|\LKeyF{8}|     & \LARGE\strut\fbox{\LKeyF{8}} \\
F9          & \verb|\LKeyF{9}|     & \LARGE\strut\fbox{\LKeyF{9}} \\
F10         & \verb|\LKeyF{10}|    & \LARGE\strut\fbox{\LKeyF{10}} \\
F11         & \verb|\LKeyF{11}|    & \LARGE\strut\fbox{\LKeyF{11}} \\
F12         & \verb|\LKeyF{12}|    & \LARGE\strut\fbox{\LKeyF{12}} \\
F13         & \verb|\LKeyF{13}|    & \LARGE\strut\fbox{\LKeyF{13}} \\
F14         & \verb|\LKeyF{14}|    & \LARGE\strut\fbox{\LKeyF{14}} \\
F15         & \verb|\LKeyF{15}|    & \LARGE\strut\fbox{\LKeyF{15}} \\
F16         & \verb|\LKeyF{16}|    & \LARGE\strut\fbox{\LKeyF{16}} \\
%
PAD0        & \verb|\LKeyPad{1}|     & \LARGE\strut\fbox{\LKeyPad{0}} \\
PAD1        & \verb|\LKeyPad{1}|     & \LARGE\strut\fbox{\LKeyPad{1}} \\
PAD2        & \verb|\LKeyPad{2}|     & \LARGE\strut\fbox{\LKeyPad{2}} \\
PAD3        & \verb|\LKeyPad{3}|     & \LARGE\strut\fbox{\LKeyPad{3}} \\
PAD4        & \verb|\LKeyPad{4}|     & \LARGE\strut\fbox{\LKeyPad{4}} \\
PAD5        & \verb|\LKeyPad{5}|     & \LARGE\strut\fbox{\LKeyPad{5}} \\
PAD6        & \verb|\LKeyPad{6}|     & \LARGE\strut\fbox{\LKeyPad{6}} \\
PAD7        & \verb|\LKeyPad{7}|     & \LARGE\strut\fbox{\LKeyPad{7}} \\
PAD8        & \verb|\LKeyPad{8}|     & \LARGE\strut\fbox{\LKeyPad{8}} \\
PAD9        & \verb|\LKeyPad{9}|     & \LARGE\strut\fbox{\LKeyPad{9}} \\
PAD10       & \verb|\LKeyPad{10}|    & \LARGE\strut\fbox{\LKeyPad{10}} \\
PAD11       & \verb|\LKeyPad{11}|    & \LARGE\strut\fbox{\LKeyPad{11}} \\
PAD12       & \verb|\LKeyPad{12}|    & \LARGE\strut\fbox{\LKeyPad{12}} \\
PAD13       & \verb|\LKeyPad{13}|    & \LARGE\strut\fbox{\LKeyPad{13}} \\
PAD14       & \verb|\LKeyPad{14}|    & \LARGE\strut\fbox{\LKeyPad{14}} \\
\end{longtable}

\clearpage
\subsection{General Keyboard}
\begin{longtable}[l]{lll}
0           & \verb|\LKey{zero}|,\verb|\LKey{0}|    & \LARGE\strut\fbox{\LKey{0}} \\
9           & \verb|\LKey{nine}|,\verb|\LKey{9}|   & \LARGE\strut\fbox{\LKey{nine}} \\
A           & \verb|\LKey{A}|      & \LARGE\strut\fbox{\LKey{A}} \\
Z           & \verb|\LKey{Z}|      & \LARGE\strut\fbox{\LKey{Z}} \\
\end{longtable}

\subsection{Keyboard Shortcuts}
\begin{longtable}[l]{lll}
Strg-A           & \verb|\LKeyStrgX{A}|       & \LARGE\strut\fbox{\LKeyStrgX{A}} \\
Ctrl-A           & \verb|\LKeyCtrlX{A}|       & \LARGE\strut\fbox{\LKeyCtrlX{A}} \\
Shift-A          & \verb|\LKeyShiftX{A}|      & \LARGE\strut\fbox{\LKeyShiftX{A}} \\
Alt-A            & \verb|\LKeyAltX{A}|        & \LARGE\strut\fbox{\LKeyAltX{A}} \\
AltGr-A          & \verb|\LKeyAltGrX{A}|      & \LARGE\strut\fbox{\LKeyAltGrX{A}} \\
Shift-Strg-A     & \verb|\LKeyShiftStrgX{A}|  & \LARGE\strut\fbox{\LKeyShiftStrgX{A}} \\
Shift-Ctrl-A     & \verb|\LKeyShiftCtrlX{A}|  & \LARGE\strut\fbox{\LKeyShiftCtrlX{A}} \\
Shift-Alt-A      & \verb|\LKeyShiftAltX{A}|   & \LARGE\strut\fbox{\LKeyShiftAltX{A}} \\
Shift-AltGr-A    & \verb|\LKeyShiftAltGrX{A}| & \LARGE\strut\fbox{\LKeyShiftAltGrX{A}} \\
Strg-Alt-A       & \verb|\LKeyStrgAltX{A}|    & \LARGE\strut\fbox{\LKeyStrgAltX{A}} \\
Strg-Alt-Entf    & \verb|\LKeyStrgAltEnt|     & \LARGE\strut\fbox{\LKeyStrgAltEnt} \\
Strg-Alt-Entf    & \verb|\LKeyReset|          & \LARGE\strut\fbox{\LKeyReset} \\
Ctrl-Alt-A       & \verb|\LKeyCtrlAltX{A}|    & \LARGE\strut\fbox{\LKeyCtrlAltX{A}} \\
Ctrl-Alt-Entf    & \verb|\LKeyCtrlAltEnt|     & \LARGE\strut\fbox{\LKeyCtrlAltEnt} \\
%
Alt-F1           & \verb|\LKeyAltF{1}|        & \LARGE\strut\fbox{\LKeyAltF{1}} \\
Alt-F2           & \verb|\LKeyAltF{2}|        & \LARGE\strut\fbox{\LKeyAltF{2}} \\
Alt-F3           & \verb|\LKeyAltF{3}|        & \LARGE\strut\fbox{\LKeyAltF{3}} \\
Alt-F4           & \verb|\LKeyAltF{4}|        & \LARGE\strut\fbox{\LKeyAltF{4}} \\
Alt-F5           & \verb|\LKeyAltF{5}|        & \LARGE\strut\fbox{\LKeyAltF{5}} \\
Alt-F6           & \verb|\LKeyAltF{6}|        & \LARGE\strut\fbox{\LKeyAltF{6}} \\
Alt-F7           & \verb|\LKeyAltF{7}|        & \LARGE\strut\fbox{\LKeyAltF{7}} \\
Alt-F8           & \verb|\LKeyAltF{8}|        & \LARGE\strut\fbox{\LKeyAltF{8}} \\
Alt-F9           & \verb|\LKeyAltF{9}|        & \LARGE\strut\fbox{\LKeyAltF{9}} \\
Alt-F10          & \verb|\LKeyAltF{10}|       & \LARGE\strut\fbox{\LKeyAltF{10}} \\
Alt-F11          & \verb|\LKeyAltF{11}|       & \LARGE\strut\fbox{\LKeyAltF{11}} \\
Alt-F12          & \verb|\LKeyAltF{12}|       & \LARGE\strut\fbox{\LKeyAltF{12}} \\
Alt-F13          & \verb|\LKeyAltF{13}|       & \LARGE\strut\fbox{\LKeyAltF{13}} \\
Alt-F14          & \verb|\LKeyAltF{14}|       & \LARGE\strut\fbox{\LKeyAltF{14}} \\
Alt-F15          & \verb|\LKeyAltF{15}|       & \LARGE\strut\fbox{\LKeyAltF{15}} \\
Alt-F16          & \verb|\LKeyAltF{16}|       & \LARGE\strut\fbox{\LKeyAltF{16}} \\
%
Strg-Alt-F1           & \verb|\LKeyStrgAltF{1}|        & \LARGE\strut\fbox{\LKeyStrgAltF{1}} \\
Strg-Alt-F2           & \verb|\LKeyStrgAltF{2}|        & \LARGE\strut\fbox{\LKeyStrgAltF{2}} \\
Strg-Alt-F3           & \verb|\LKeyStrgAltF{3}|        & \LARGE\strut\fbox{\LKeyStrgAltF{3}} \\
Strg-Alt-F4           & \verb|\LKeyStrgAltF{4}|        & \LARGE\strut\fbox{\LKeyStrgAltF{4}} \\
Strg-Alt-F5           & \verb|\LKeyStrgAltF{5}|        & \LARGE\strut\fbox{\LKeyStrgAltF{5}} \\
Strg-Alt-F6           & \verb|\LKeyStrgAltF{6}|        & \LARGE\strut\fbox{\LKeyStrgAltF{6}} \\
Strg-Alt-F7           & \verb|\LKeyStrgAltF{7}|        & \LARGE\strut\fbox{\LKeyStrgAltF{7}} \\
Strg-Alt-F8           & \verb|\LKeyStrgAltF{8}|        & \LARGE\strut\fbox{\LKeyStrgAltF{8}} \\
Strg-Alt-F9           & \verb|\LKeyStrgAltF{9}|        & \LARGE\strut\fbox{\LKeyStrgAltF{9}} \\
Strg-Alt-F10          & \verb|\LKeyStrgAltF{10}|       & \LARGE\strut\fbox{\LKeyStrgAltF{10}} \\
Strg-Alt-F11          & \verb|\LKeyStrgAltF{11}|       & \LARGE\strut\fbox{\LKeyStrgAltF{11}} \\
Strg-Alt-F12          & \verb|\LKeyStrgAltF{12}|       & \LARGE\strut\fbox{\LKeyStrgAltF{12}} \\
Strg-Alt-F13          & \verb|\LKeyStrgAltF{13}|       & \LARGE\strut\fbox{\LKeyStrgAltF{13}} \\
Strg-Alt-F14          & \verb|\LKeyStrgAltF{14}|       & \LARGE\strut\fbox{\LKeyStrgAltF{14}} \\
Strg-Alt-F15          & \verb|\LKeyStrgAltF{15}|       & \LARGE\strut\fbox{\LKeyStrgAltF{15}} \\
Strg-Alt-F16          & \verb|\LKeyStrgAltF{16}|       & \LARGE\strut\fbox{\LKeyStrgAltF{16}} \\
Ctrl-Alt-F1           & \verb|\LKeyCtrlAltF{1}|        & \LARGE\strut\fbox{\LKeyCtrlAltF{1}} \\
Ctrl-Alt-F2           & \verb|\LKeyCtrlAltF{2}|        & \LARGE\strut\fbox{\LKeyCtrlAltF{2}} \\
Ctrl-Alt-F3           & \verb|\LKeyCtrlAltF{3}|        & \LARGE\strut\fbox{\LKeyCtrlAltF{3}} \\
Ctrl-Alt-F4           & \verb|\LKeyCtrlAltF{4}|        & \LARGE\strut\fbox{\LKeyCtrlAltF{4}} \\
Ctrl-Alt-F5           & \verb|\LKeyCtrlAltF{5}|        & \LARGE\strut\fbox{\LKeyCtrlAltF{5}} \\
Ctrl-Alt-F6           & \verb|\LKeyCtrlAltF{6}|        & \LARGE\strut\fbox{\LKeyCtrlAltF{6}} \\
Ctrl-Alt-F7           & \verb|\LKeyCtrlAltF{7}|        & \LARGE\strut\fbox{\LKeyCtrlAltF{7}} \\
Ctrl-Alt-F8           & \verb|\LKeyCtrlAltF{8}|        & \LARGE\strut\fbox{\LKeyCtrlAltF{8}} \\
Ctrl-Alt-F9           & \verb|\LKeyCtrlAltF{9}|        & \LARGE\strut\fbox{\LKeyCtrlAltF{9}} \\
Ctrl-Alt-F10          & \verb|\LKeyCtrlAltF{10}|       & \LARGE\strut\fbox{\LKeyCtrlAltF{10}} \\
Ctrl-Alt-F11          & \verb|\LKeyCtrlAltF{11}|       & \LARGE\strut\fbox{\LKeyCtrlAltF{11}} \\
Ctrl-Alt-F12          & \verb|\LKeyCtrlAltF{12}|       & \LARGE\strut\fbox{\LKeyCtrlAltF{12}} \\
Ctrl-Alt-F13          & \verb|\LKeyCtrlAltF{13}|       & \LARGE\strut\fbox{\LKeyCtrlAltF{13}} \\
Ctrl-Alt-F14          & \verb|\LKeyCtrlAltF{14}|       & \LARGE\strut\fbox{\LKeyCtrlAltF{14}} \\
Ctrl-Alt-F15          & \verb|\LKeyCtrlAltF{15}|       & \LARGE\strut\fbox{\LKeyCtrlAltF{15}} \\
Ctrl-Alt-F16          & \verb|\LKeyCtrlAltF{16}|       & \LARGE\strut\fbox{\LKeyCtrlAltF{16}} \\
\end{longtable}

\clearpage
\subsection{Mouse Buttons (Three-Button Mice)}
\begin{longtable}[l]{lll}
Empty        & \verb|\LMouseEmpty|   & \LARGE\strut\fbox{\LMouseEmpty} \\
No           & \verb|\LMouseN|       & \LARGE\strut\fbox{\LMouseN} \\
Left         & \verb|\LMouseL|       & \LARGE\strut\fbox{\LMouseL} \\
Middle       & \verb|\LMouseM|       & \LARGE\strut\fbox{\LMouseM} \\
Right        & \verb|\LMouseR|      & \LARGE\strut\fbox{\LMouseR} \\
LeftRight    & \verb|\LMouseLR|      & \LARGE\strut\fbox{\LMouseLR} \\
\end{longtable}

\subsection{Mouse Buttons (Two-Button Mice)}
\begin{longtable}[l]{lll}
Empty        & \verb|\LMouseIIEmpty|   & \LARGE\strut\fbox{\LMouseIIEmpty} \\
No           & \verb|\LMouseIIN|       & \LARGE\strut\fbox{\LMouseIIN} \\
Left         & \verb|\LMouseIIL|       & \LARGE\strut\fbox{\LMouseIIL} \\
Right        & \verb|\LMouseIIR|       & \LARGE\strut\fbox{\LMouseIIR} \\
LeftRight    & \verb|\LMouseIILR|      & \LARGE\strut\fbox{\LMouseIILR} \\
\end{longtable}


\clearpage
\section{Linux Biolinum Keyboard Glyphs}\small\tt
\label{KeyboardGlyphs}
\renewcommand\DeclareTextGlyphY[3]{\makebox[2.5cm]{\LARGE\strut\fbox{\biolinumKeyGlyph{#2}}} #2\\}%
\catcode`\_=12%
\begin{multicols}{2}
\par\noindent
\input{LinBiolinum_K}
\end{multicols}


\clearpage
\section{Selected Libertine Initials}\tt
\label{InitialGlyphs}
\renewcommand\DeclareTextGlyphY[3]{\makebox[2.0cm]{\LARGE\strut\fbox{\libertineInitialGlyph{#2}}} #2\\}%
\catcode`\_=12%
\begin{multicols}{4}
\par\noindent
\DeclareTextGlyphY{LinLibertine_I}{zero}{48}
\DeclareTextGlyphY{LinLibertine_I}{one}{49}
\DeclareTextGlyphY{LinLibertine_I}{two}{50}
\DeclareTextGlyphY{LinLibertine_I}{three}{51}
\DeclareTextGlyphY{LinLibertine_I}{four}{52}
\DeclareTextGlyphY{LinLibertine_I}{five}{53}
\DeclareTextGlyphY{LinLibertine_I}{six}{54}
\DeclareTextGlyphY{LinLibertine_I}{seven}{55}
\DeclareTextGlyphY{LinLibertine_I}{eight}{56}
\DeclareTextGlyphY{LinLibertine_I}{nine}{57}
\DeclareTextGlyphY{LinLibertine_I}{A}{65}
\DeclareTextGlyphY{LinLibertine_I}{B}{66}
\DeclareTextGlyphY{LinLibertine_I}{C}{67}
\DeclareTextGlyphY{LinLibertine_I}{D}{68}
\DeclareTextGlyphY{LinLibertine_I}{E}{69}
\DeclareTextGlyphY{LinLibertine_I}{F}{70}
\DeclareTextGlyphY{LinLibertine_I}{G}{71}
\DeclareTextGlyphY{LinLibertine_I}{H}{72}
\DeclareTextGlyphY{LinLibertine_I}{I}{73}
\DeclareTextGlyphY{LinLibertine_I}{J}{74}
\DeclareTextGlyphY{LinLibertine_I}{K}{75}
\DeclareTextGlyphY{LinLibertine_I}{L}{76}
\DeclareTextGlyphY{LinLibertine_I}{M}{77}
\DeclareTextGlyphY{LinLibertine_I}{N}{78}
\DeclareTextGlyphY{LinLibertine_I}{O}{79}
\DeclareTextGlyphY{LinLibertine_I}{P}{80}
\DeclareTextGlyphY{LinLibertine_I}{Q}{81}
\DeclareTextGlyphY{LinLibertine_I}{R}{82}
\DeclareTextGlyphY{LinLibertine_I}{S}{83}
\DeclareTextGlyphY{LinLibertine_I}{T}{84}
\DeclareTextGlyphY{LinLibertine_I}{U}{85}
\DeclareTextGlyphY{LinLibertine_I}{V}{86}
\DeclareTextGlyphY{LinLibertine_I}{W}{87}
\DeclareTextGlyphY{LinLibertine_I}{X}{88}
\DeclareTextGlyphY{LinLibertine_I}{Y}{89}
\DeclareTextGlyphY{LinLibertine_I}{Z}{90}
\end{multicols}


% To generate a table of all the glyphs, uncomment the following lines:
%
%\begin{multicols}{2}
%\par\noindent
%\input{LinLibertine_I}
%\end{multicols}

\clearpage
\section{Implementation Notes}\rm
\label{impl}

\subsection{Aims}

Modern OpenType and TrueType fonts are not directly usable with traditional
typesetting engines such as \LaTeX\ or pdf\LaTeX.  On the other hand, many documents
that use traditional font-selection mechanisms cannot be processed by
emerging new technologies such as xe\LaTeX\ and lua\LaTeX.  The primary aim
of the \texttt{libertine} package is, as much as possible, to allow documents
to use Linux Libertine and Biolinum fonts compatibly with \emph{all} current \LaTeX\
engines.  Another aim is maintainability: it should be possible to update
the package easily when updated fonts become available.

\subsection{The Fonts}

OpenType Linux Libertine and Biolinum fonts (with \texttt{otf} extensions) may be downloaded from
\url{http://sourceforge.net/projects/linuxlibertine/files/linuxlibertine/}.
There are a few problems with the current versions of the fonts (5.3.0).
\begin{itemize}
\item
Currently, there is no bold-italic variant of the Biolinum family; an \emph{ad hoc}
solution is to use \texttt{fontforge} to generate an artificially slanted
version of the bold variant.  Note that the most recent version of \texttt{fontforge}
must be used on Biolinum fonts; an earlier version will generate fonts with incorrect \texttt{ex}-height.
\item Slanted (oblique) variants are not available
from the upstream site.  These could be
generated easily but we have decided not to attempt to support slanted
variants for the fonts; the italic (or fake-italic) variants will be silently substituted.
\item
The bold-italic variant of the Libertine family is missing
several ligatures; the ligatures would be taken from the
regular-weight italic variant, which is unacceptable.
Michael Sharpe (\url{msharpe@ucsd.edu}) has generated
the missing glyphs  (\emph{\bfseries fl, ffl, ffi})  and added them to the \texttt{otf}
file.

\item Currently, Libertine Monospaced does not have bold, italic or bold-italic
variants; \texttt{fontforge} has been used to generate artificially emboldened and/or slanted
variants.

\item
When several of the fonts are opened in \texttt{fontforge}, warning
messages are generated about errors in the glyph programs.  Some of
these are sufficient to cause failures or even crashes when conversion
to \typeone\ format is attempted using \texttt{cfftot1}. Michael
Sharpe has corrected the most serious of these.  In some cases,
\texttt{fontforge}
has been used to convert the format, as it is less sensitive than \texttt{cfftot1}
to faulty glyph programs.
\end{itemize}

In some \TeX\ distributions, the OpenType and \typeone\ fonts are installed as system fonts,
and xe\LaTeX\ or lua\LaTeX\ users may attempt to select the OpenType fonts directly by
their Postscript FontName.  If \typeone\ versions with the \emph{same} FontName
have been installed, the latter may be selected by the system font-selection
mechanism.  To avoid this, it is appropriate to modify the
FontNames of the \texttt{otf} fonts before converting to \typeone\
format (but not \emph{distribute} these re-named \texttt{otf} fonts).
The \typeone\  Libertine and Biolinum fonts distributed in this
package have had the \texttt{O} (for Opentype) in their FontNames replaced by
\texttt{T}
(for Type~\liningnums{1}) using \texttt{fontforge}.  This font-renaming
must be done \emph{before} generating the \LaTeX-support files, or else
\texttt{dvi2ps} will fail.


\subsection{Generation of Support Files}

The \texttt{otftotfm} tool of the \texttt{lcdftypetools}
package and the \texttt{autoinst} script of the \texttt{fontools} package are convenient tools
for generating \LaTeX\ support files for OpenType families.
To generate a \texttt{texmf} tree for the \texttt{libertine}
package on a Unix-like system, one puts all the \texttt{otf} files to be supported for \LaTeX\ or
pdf\LaTeX\footnote{Currently, all of the OpenType fonts except the Keyboard font are supported
for \LaTeX\ and pdf\LaTeX.} into a directory, creates a \texttt{texmf} sub-directory  and executes
\begin{verbatim}
    autoinst -target=./texmf -encoding=OT1,T1,LY1,TS1 \
      -vendor=public -typeface=libertine -noupdmap    \
      -noswash -notitling -noornaments                \
      *.otf
\end{verbatim}
Then move to the \texttt{texmf} directory and do
\begin{verbatim}
    rm -rf fonts/pl fonts/vpl fonts/truetype fonts/type42
    mv fonts/enc/dvips/public fonts/enc/dvips/libertine
    mv fonts/map/dvips/public fonts/map/dvips/libertine
\end{verbatim}
to delete irrelevant sub-directories and re-name directories as required by
TeXLive.

A few additional steps are needed.

\subsubsection{Renaming of the Encoding Files}
\texttt{otftotfm} generates encoding files with filenames of the form \verb|a_xxxxxx|;
to avoid filename conflicts with other packages, the files have been re-named
to have a distinctive prefix using the
command
\begin{verbatim}
    rename_enc libertine lbtn
\end{verbatim}
executed in the \texttt{texmf} directory, where
\verb|rename_enc| is a PERL script in
\begin{verbatim}
    doc/fonts/libertine
\end{verbatim}
Then in \verb|fonts/map/dvips/libertine|,
the map files can be concatenated into a single file \verb|libertine.map| and
all instances of \verb|a_| changed to \verb|lbtn_|; the
original \verb|map| files have been deleted.

\subsubsection{Installation of the Fonts}

The \texttt{otf} files after corrections (but before re-naming) are
installed into the \texttt{texmf} tree in the following sub-directory:
\begin{verbatim}
    fonts/OpenType/public/libertine/
\end{verbatim}
The \texttt{autoinst} script will normally use \texttt{cfftot1}
to create \verb|pfb| files with appropriate internal names
and filenames; but if more than one font family has been processed
or if \texttt{cfftot1} runs into trouble, this may not happen.  In that
case, one must do the conversion font-by-font using either \texttt{cfftot1}
or \texttt{fontforge}; the appropriate internal names and filenames are as specified in
\verb|libertine.map|.
The \texttt{pfb} files are installed into the \texttt{texmf} tree in the following
sub-directory:
\begin{verbatim}
    fonts/type1/public/libertine/
\end{verbatim}

\subsubsection{The \texttt{fd} Files}
The \texttt{autoinst} script generates a large number of files with
\verb|.fd| extensions in the \verb|tex/latex/libertine/| directory.
Recent versions will generate ``silent substitution''
rules for mapping \texttt{sl} to \texttt{it} and \texttt{bx} to \texttt{b};
if not, these have to be added by hand.

\subsubsection{The \texttt{sty} Files}
The \texttt{autoinst} script generates files with \verb|.sty| extensions
in the \verb|tex/latex/libertine/| directory for each of the font families;
but these are useless for xe\LaTeX\ and lua\LaTeX\ users and have been deleted.
A \verb|libertine.sty| file has been generated ``by hand'' and is discussed in Section~\ref{sty}.

\subsection{\texttt{libertine.sty}}
\label{sty}
This file implements the support for both \typeone\ and OpenType usage; the choice
is initially determined by the processing engine, but as some xe\LaTeX\  and lua\LaTeX\
users may prefer to avoid \texttt{fontspec}, a \texttt{type1} (or \texttt{nofontspec}) option
is provided to change this.

The \verb\...@scale\ commands are invoked in the \verb\fd\ files or when
specifying fonts with \texttt{fontspec}; only the scale factors for Biolinum and
Libertine Mono are adjustable using option parameters.

If the \texttt{sfdefault} option has been used, the \verb|\familydefault|
is set to the \emph{current} value of \verb|\sfdefault| (with no change to
\verb|\rmdefault|).

The use of \verb|\newfontfamily| rather than \verb|\addfontfeatures| avoids
problems in the implementation of the latter for some fonts (including, unfortunately,
Libertine).

For the Mono and Keyboard font families, the Ligature and SmallCap features must be turned off.

Commands to switch locally to oldstyle/lining/proportional/tabular numbers
are defined; the definitions of \verb|\oldstylenums| must deal with possible
pre-existing definitions.

To implement the \verb|\...Glyph| commands, it is necessary to, essentially, iterate
through all the \emph{defined} glyphs in the relevant OpenType font.  This is implemented
by creating files \verb|LinLibertine_R.tex|, \verb|\LinBiolinum_R.tex|,
\verb|LinBiolinum_K.tex| and \verb|LinLibertine_I.tex| which declare the glyph name (when available),
unicode code point, and glyph index for every defined glyph.
These files are created by using \texttt{fontforge} to generate
a ``glyph map'' file (extension \verb|.g2n|) for the relevant font and then the small \texttt{C} program
\verb|doc/fonts/libertine/g2ntotex.c| will convert this into the required \verb|.tex| file.

The final step in \verb|libertine.sty| is to remove all default font features
in \texttt{fontspec} in case other fonts will be activated by the user.

\subsection{Additional \texttt{sty} Files}
The \verb|tex/latex/libertine/| directory also contains three ``front-end'' files
\verb|libertineotf.sty|, \verb|libertine-type1.sty|, and \verb|biolinum-type1.sty|,
which provide partial compatibility with obsolete packages, primarily for legacy
documents.

\end{document}
